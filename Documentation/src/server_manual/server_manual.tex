\documentclass[12pt]{article}

\usepackage[german]{babel}
\usepackage[T1]{fontenc} % for typewriter braces
\usepackage[utf8x]{inputenc}
\usepackage{amsmath}
\usepackage{graphicx}
\usepackage[colorinlistoftodos]{todonotes}
% Simple directory trees
\usepackage{dirtree}
\usepackage{nameref}
\usepackage{listings}
\usepackage{tabularx}
\usepackage{wasysym}

% \meta command
\usepackage{textcomp}
\usepackage{hyperref}
\usepackage{amssymb}
\providecommand\meta[1]{\textlangle{\itshape #1\/}\textrangle}

\lstdefinelanguage{incomplete-Ini}
{
basicstyle=\ttfamily\small,
columns=fullflexible,
%morecomment=[s][\color{Orchid}\bfseries]{[}{]},
morecomment=[l]{\#},
morecomment=[l]{;},
commentstyle=\color{gray}\ttfamily,
morekeywords={},
otherkeywords={=},
keywordstyle={\color{green}\bfseries}
}

\graphicspath{ {../common/} }


% Make \input search ./ and ../common/
\makeatletter
\def\input@path{{./}{../common/}}
%or: \def\input@path{{/path/to/folder/}{/path/to/other/folder/}}
\makeatother

\newenvironment{fig}[1][htb]
{
\begin{figure}[#1]
\centering
}
{
\end{figure}
}

\newcommand{\img}[4][]{
\begin{fig}
\includegraphics[#1]{#2}
\caption{#3}
\label{img:#4}
\end{fig}
}

%\newcommand{\img}[2][htb]{%
%\begin{figure}[#1]
%    \centering
%    \includegraphics{#2}
%\end{figure}%
%}

% Define \Begin command
\newcommand{\Begin}{
\begin{document}

\begin{titlepage}

    \newcommand{\HRule}{\rule{\linewidth}{0.5mm}} % Defines a new command for the horizontal lines, change thickness here

    \center % Center everything on page

    % Heading
    \textsc{\LARGE GFOS Innovationsaward 2018}\\[1.5cm]
    \textsc{\Large Helmholtz-Gymnasium Essen}\\[0.5cm]
    \textsc{\large Dokumentation}\\[0.5cm]

    % Main title
    \HRule \\[0.4cm]
    { \huge \bfseries \thetitle}\\[0.4cm]
    \HRule \\[1.5cm]

    % Authors
    \begin{minipage}{0.4\textwidth}
        \begin{flushleft}
            \large
            \emph{Autoren:}\\
            Julian Koch\\
            Mats Holtbecker
        \end{flushleft}
    \end{minipage}
    ~
    \begin{minipage}{0.4\textwidth}
        \begin{flushright}
            \large
            \emph{Aufsicht:} \\
            Dennis Gro"skamp\\
            \hfill
            % Empty line
        \end{flushright}
    \end{minipage}\\[2cm]

    % If you don't want a supervisor, uncomment the two lines below and remove the section above
    %\Large \emph{Author:}\\
    %John \textsc{Smith}\\[3cm] % Your name

    % Date
    {\large \today}\\[2cm]

    % GFOS Logo
    \includegraphics[scale=0.7]{gfos_logo_edited.jpg}\\[1cm]

    %----------------------------------------------------------------------------------------

    \vfill
    % Fill the rest of the page with whitespace

\end{titlepage}


\newpage
\tableofcontents
\newpage
}

% Define \End command
\newcommand{\End}{%
\end{document}%
}


\newcommand*{\thetitle}{ZNews Server\\ Bedienungsanleitung}

\Begin

%% We don't need an abstract in this page
%% We'll add that to Dokumentation.pdf
%\begin{abstract}
%    ZNews besteht weitestgehend aus zwei sich gegen"uberstehenden
%    Komponenten: dem Server und dem Webinterface.
%    In diesem Dokument wird die Funktionsweise des Servers
%    dargestellt und erl"autert.
%\end{abstract}

%\newpage
%\tableofcontents
%\newpage

\section{Installation}

ZNews Server wird als archivierte Datei geliefert.
Nach deren Download kann sie mit einem Archivierungsprogramm
wie WinRAR oder 7-Zip ausgepackt werden.

\begin{enumerate}
    \item Extrahieren Sie nach deren Download die Datei \emph{ZNews.zip}.
    Die Struktur des Hauptordners ist in Abb.~\ref{fig:dirtree1} dargestellt.
    \item Starten Sie den Server mit dem Startskript\footnote{%
    Windows: \emph{start.bat}, Mac: \emph{start.command}}.
    \item Nach Erscheinen der Nachricht ``Server started'' (o."a.),
    Schreiben Sie ``end'' und dr"ucken Sie \meta{Enter}.
    \item Nach Beenden des Programms sind die in Abb.~\ref{fig:dirtree2}
    gezeigten Dateien und Ordner neu erstellt worden.
    Konfigurieren Sie den Server wie in Abschnitt~\ref{sec:config} ``\nameref{sec:config}'' beschrieben.
    \item Sie k"onnen nun jederzeit den Server mit dem Startskript starten.
\end{enumerate}

\begin{figure}[htb]
    \dirtree{%
    .1 /.
    .2 server.jar.
    .2 start.bat.
    .2 start.command.
    .2 docs/.
    .3 de/.
    .4 Bedienungsanleitung.pdf.
    .4 Server Bedienungsanleitung.pdf.
    .4 Dokumentation.pdf.
    }
    \caption{\label{fig:dirtree1}Aufbau des heruntergeladenen Archivs}
\end{figure}

\begin{figure}[htb]
    \dirtree{%
    .1 /.
    .2 logs/.
    .2 static\textunderscore web/.
    .2 auth.json.
    .2 config.properties.
    .2 newsletters.json.
    .2 registrations.json.
    .2 \ldots.
    }
    \caption{\label{fig:dirtree2}Neu erstellte Dateien und Ordner}
\end{figure}

\pagebreak
\section{Konfiguration}%
\label{sec:config}

ZNews Server beinhaltet viele Konfigurationsm"oglichkeiten,
allesamt in \emph{config.properties}.
Dieser Abschnitt zeigt die wichtigsten Konfigurationen
und bietet au"serdem eine vollst"andige Referenz am Ende.

\subsection{Wichtige Konfigurationsschl"ussel}

\subsubsection{Der Port} \label{sec:config:port}

\begin{align*}
    \emph{Schl"ussel: } & \lstinline{port}\\
    \emph{Standardwert: } & \lstinline{8080}
\end{align*}
%% Consider this other, non-centered version:
%\hspace*{2cm}\emph{Schl"ussel: } \lstinline{port}\\
%\hspace*{2cm}\emph{Standardwert: } \lstinline{8080}
%\vspace*{0.2cm}

ZNews Server beansprucht einen Port, um eingehende
Anfragen bearbeiten zu k"onnen. Standardm"a"sig ist
\emph{8080} eingestellt, jedoch ist es empfohlen,
diesen Wert zu "andern.

\subsubsection{E-Mail-Einstellungen}

ZNews Server muss dazu in der Lage sein, E-Mails
zu versenden. Dazu muss z.B. ein SMTP-Server eingestellt
werden, durch welchen die E-Mails geschickt werden.

Im Folgenden wird ein Gmail-Account mit der
E-Mail-Adresse \emph{john.smith@gmail.com} und
dem Passwort \emph{12345} verwendet:

\begin{lstlisting}[]%@formatter:off
email.protocol=smtp
email.port=465
email.host=smtp.gmail.com
email.auth=true
email.auth.usr=john.smith@gmail.com
email.auth.pw=12345
\end{lstlisting}%@formatter:on

F"ur die Verwendung von anderen Servern, konsultieren Sie
bitte deren spezifischen Anleitungen.

\subsubsection{Log-Level}

\begin{align*}
    \emph{Schl"ussel: } & \lstinline{log.filter}\\
    \emph{Standardwert: } & \lstinline{out}\\
    \emph{M"ogliche Werte: } & \lstinline{dev|debug|out|warn|err|fatal}
\end{align*}

Das Log-Level spezifiziert, welche Logging-Ausgaben tats"achlich
ausgegeben werden. Zum Beispiel werden bei \lstinline{warn}
nur Warnungen, Fehler (\lstinline{err}) und schwerwiegende Fehler
(\lstinline{fatal}) ausgegeben.

\subsection{Referenz}

Eine vollst"andige Referenz der Konfiguration
findet sich Tabelle~\ref{table:config-ref}
und Abbildung~\ref{fig:config-std}.

\section{Befehle}

ZNews Server unterst"utzt eine Reihe von Befehlen,
die in die Konsole eingegeben werden k"onnen.
Um einen Befehl auszuf"uhren, schreiben Sie ihn
in die Eingabeaufforderung und best"atigen Sie
mit \meta{Enter}.

\subsection{Befehlsreferenz}

\subsubsection{end}

\hspace*{2cm}\emph{Syntax: } \lstinline{end}
\vspace*{0.2cm}

\emph{end} stoppt den Server, speichert alle Daten
und komprimiert die Log-Datei in einer Gzip-Datei.

Das Beenden des Servers kann einige Augenblicke
in Anspruch nehmen; die letzte Zeile in der
Konsole sollte ``(End)'' zeigen.

\subsubsection{restart}

\hspace*{2cm}\emph{Syntax: } \lstinline{restart}
\vspace*{0.2cm}

Startet den Server neu. "Ahnlich zu \emph{end}
werden alle Daten gespeichert, jedoch wird
das Logging-System nicht beendet, we"shalb
die Log-Datei nicht komprimiert wird.

\subsubsection{reset caches}

\hspace*{2cm}\emph{Syntax: } \lstinline{reset caches} oder \lstinline{rs}
\vspace*{0.2cm}

Leert (\emph{purge}) den Cache, sofern dieser
in der Konfiguration eingeschaltet ist.

\subsubsection{addadmin}

\hspace*{2cm}\emph{Syntax: } \lstinline{addadmin} \meta{email} \meta{name} \meta{password}
\vspace*{0.2cm}

Erstellt einen neuen Administrator-Account mit
der E-Mail Adresse \meta{email}, dem Namen \meta{name}
und dem Passwort \meta{password}.


\section{Troubleshooting}

\subsection[``Failed to bind port'']{Beim Starten des Servers erscheint die Nachricht ``Failed to bind port'' mit zugehörigem StackTrace.}
\begin{enumerate}
    \item Überprüfen Sie, ob der Server noch an einer anderen Stelle gestartet ist. Nutzen Sie zur Not
    einen Task Manager, um eine zweite Server-Instanz zu finden, oder starten Sie den PC / die
    virtuelle Maschine neu.

    \emph{Hinweis: Es ist möglich, mehrere Server-Instanzen gleichzeitig laufen zu lassen, ändern Sie
    dafür den Port in der Konfigurationsdatei eines der Server.}
    \item Läuft keine zweite Server-Instanz, so ist davon auszugehen, dass ein anderes Programm den
    von Feedback Server genutzten Port bereits belegt. Ist dies der Fall, haben Sie drei
    Möglichkeiten:
    \begin{enumerate}
        \item Schließen Sie das Programm und stellen Sie sicher, dass es nicht automatisch wieder gestartet wird.
        \item Konfigurieren Sie das Programm so, einen anderen Port zu nutzen. Konsultieren Sie ggf. das dem Programm zugehörige Benutzerhandbuch
        \item Konfigurieren Sie Feedback Server so, einen anderen Port zu nutzen\footnote{Siehe Abschnitt~\ref{sec:config:port}, ``\nameref{sec:config:port}''}.
    \end{enumerate}
\end{enumerate}

\subsection[Mac: Fehler beim Ausf"uhren von start.command]{(Mac) Ich kann die start.command-Datei nicht ausführen (keine Berechtigung).}
\begin{enumerate}
    \item Melden Sie sich mit administrativen Rechten an.
    \item Sollte das nicht funktionieren, öffnen Sie Terminal, navigieren Sie mit Hilfe des Kommandos
    \emph{cd} in den Ordner der Datei (konsultieren Sie ggf. das Internet) und schreiben Sie \emph{sudo sh
    start.command}. Bei Bestätigung mit Enter sollte sich der Server starten. Unter Umständen werden Sie
    vorher nach Ihrem Passwort gefragt.
\end{enumerate}


\begin{table}[p]
    \centering
    \begin{tabularx}{\textwidth}{|X|X|X|}
        \hline
        \textbf{Schl"ussel} = Standard & M"ogliche Werte & Beschreibung\\ \hline
        \textbf{port} = 8080 & 1-65535 & Der von ZNews Server genutzte Port\\ \hline
        \textbf{err\textunderscore docs.404} = error/ 404notfound.html & any file path & Das Dokument, dass bei Anfragen, die mit 404 beantwortet werden, gesendet wird\\ \hline
        \textbf{data.method} = file & file|mysql & Die Methode, mit welcher Daten gespeichert werden\\ \hline
        \textbf{cache.enabled} = true & true|false & Ob der Cache von \emph{static\textunderscore web/} aktiviert werden soll\\ \hline
        \textbf{cache.size} = 32 & $\in\mathbb{N}$ & Die maximale Gr"o"se des Caches\\ \hline
        \textbf{pretty-print-json} = false & true|false & Erh"oht die Lesbarkeit der JSON-Dateien\\ \hline
        \textbf{email.protocol} = smtp & smtp|smtps|tls & Das Email-Server-Protokoll\\ \hline
        \textbf{email.port} = 465 & 1-65535 & Der Email-Server-Port\\ \hline
        \textbf{email.host} = smtp.gmail.com & Die Email-Server-Adresse\\ \hline
        \textbf{email.from} = noreply@ znews.de & E-Mail Adresse & Die E-Mail Adresse, von welcher alle E-Mails versendet werden\\ \hline
        \textbf{email.auth} = true & true|false & Ob sich ZNews Server beim Email-Server authentifizieren muss\\ \hline
        \textbf{email.auth.usr} & String & Beim Email-Server g"ultiger Login\\ \hline
        \textbf{email.auth.pw} & String & Beim Email-Server g"ultiges Password\\ \hline
        \textbf{email.debug} = false & true|false & Ob Debug-Output beim Email-Versenden eingeschaltet werden soll\\ \hline
        % TODO: Add all config values to reference
    \end{tabularx}
    \caption{\label{table:config-ref}Konfigurationsreferenz}
\end{table}

\begin{figure}[p]
    \centering
    \begin{lstlisting}[frame=single,language=incomplete-Ini]%formatter:off

port=8080

err_docs.404=error/404notfound.html

data.method=file

cache.enabled=true
cache.size=32

pretty-print-json=false

email.protocol=smtp
email.port=465
email.host=smtp.gmail.com
email.from=noreply@znews.de
email.auth=true
email.auth.usr={ACCOUNT_LOGIN}
email.auth.pw={ACCOUNT_PASSWORD}
email.debug=false

email.templates.double-opt-in=email_templates/\
        double_opt_in
email.templates.new-newsletter=email_templates/\
        new_newsletter

log.filter=out
log.format.dev=[{date}] [{threadName}:{threadId}\
        /{level} @ {methodName}/{typeSimpleName}/\
        {fileName}:{line}] {msg}
log.format.dev.date-format=HH:mm:ss:SSS
log.format.debug=[{date}] [{threadName}:{threadId}\
        /{level}] {msg}
log.format.fatal=FATAL [{date}] [{threadName}:\
        {threadId}/{level} @ {methodName}/{typeName}\
        /{fileName}:{line}] {msg}
log.out.files-gzip.dir=logs/
log.out.files-gzip.auto-flush=true
log.separate-thread=true
    \end{lstlisting}%formatter:on
    \caption{\label{fig:config-std}Standard Konfigurationsdatei}
\end{figure}


%Comments can be added to the margins of the document using the \t%odo{Here's a comment in the margin!} tod%o command, as shown in the example on the right. You can also add inline comments too:
%
%\to%do[inline, color=green!40]{This is an inline comment.}


\End
