ZNews Server wird als archivierte Datei ausgeliefert.
Nach deren Download kann sie mit einem Archivierungsprogramm
wie WinRAR oder 7-Zip ausgepackt werden.

\begin{enumerate}
    \item Extrahieren Sie nach deren Download die Datei \emph{ZNews.zip}.
    Die Struktur des Hauptordners ist in Abb.~\ref{fig:dirtree1} dargestellt.
    \item Starten Sie den Server mit dem Startskript\footnote{%
    Windows: \emph{start.bat}, Mac: \emph{start.command}}.
    \item Nach Erscheinen der Nachricht ``Server started'' (o."a.),
    Schreiben Sie ``end'' und dr"ucken Sie \meta{Enter}.
    \item Nach Beenden des Programms sind die in Abb.~\ref{fig:dirtree2}
    gezeigten Dateien und Ordner neu erstellt worden.
    Konfigurieren Sie den Server wie in Abschnitt~\ref{sec:config} ``\nameref{sec:config}'' beschrieben.
    \item Sie k"onnen nun jederzeit den Server mit dem Startskript starten.
\end{enumerate}

\begin{figure}[htb]
    \dirtree{%
    .1 /.
    .2 README.txt.
    .2 server.jar.
    .2 start.bat.
    .2 start.command.
    .2 docs/.
    .3 de/.
    .4 Bedienungsanleitung.pdf.
    .4 Server\textunderscore Bedienungsanleitung.pdf.
    .4 Dokumentation.pdf.
    }
    \caption{\label{fig:dirtree1}Aufbau des heruntergeladenen Archivs}
\end{figure}

\begin{figure}[htb]
    \dirtree{%
    .1 /.
    .2 config.properties.
    .2 auth.json.
    .2 newsletters.json.
    .2 registrations.json.
    .2 logs/.
    .2 static\textunderscore web/.
    .2 \ldots.
    }
    \caption{\label{fig:dirtree2}Neu erstellte Dateien und Ordner}
\end{figure}
