Wenn Sie einen neuen Newsletter erstellen bzw.
einen bereits bestehenden bearbeiten,
geschieht dies über unseren umfangreichen Artikel-Editor,
welcher in Abb. \ref{img:article-editor} gezeigt wird.

\img[scale=.3]{img/article_editor}{Artikel-Editor}{article-editor}

Er bietet die Möglichkeiten,
den Titel des Artikels (sogar nach Veröffentlichung!)
über das oberste Textfeld zu ändern,
diesem Themen mithilfe des untersten Textfeldes zuzuordnen
(Themen werden mit \meta{Enter} bestätigt),
und, natürlich,
den Inhalt über das Hauptfeld zu ändern.
Dazu bieten wir einen von QuillJS gepowerten RichText-Editor
mit Bilder-, Video- und Formelunterstützung und Vielem mehr.

Sie können Ihre Arbeit mithilfe des ``SAVE''-Buttons speichern.
Dabei werden Sie gewarnt,
wenn Sie vergessen haben,
einen Titel anzugeben,
oder bis jetzt noch nie verwendete Themen benutzen
(siehe Abb. \ref{img:warn-unknown-tag}).

\img[scale=.3]{img/warn_unknown_tag.png}{Warnung beim Speichern unbekannter Themen}{warn-unknown-tag}
