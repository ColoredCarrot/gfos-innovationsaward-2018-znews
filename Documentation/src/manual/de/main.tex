%%%%%%%%%%%%%%%%%%%%%%%%%%%%%%%%%%%%%%%%%
% University Assignment Title Page
% LaTeX Template
% Version 1.0 (27/12/12)
%
% This template has been downloaded from:
% http://www.LaTeXTemplates.com
%
% Original author:
% WikiBooks (http://en.wikibooks.org/wiki/LaTeX/Title_Creation)
%
% License:
% CC BY-NC-SA 3.0 (http://creativecommons.org/licenses/by-nc-sa/3.0/)
%
% Instructions for using this template:
% This title page is capable of being compiled as is. This is not useful for
% including it in another document. To do this, you have two options:
%
% 1) Copy/paste everything between \begin{document} and \end{document}
% starting at \begin{titlepage} and paste this into another LaTeX file where you
% want your title page.
% OR
% 2) Remove everything outside the \begin{titlepage} and \end{titlepage} and
% move this file to the same directory as the LaTeX file you wish to add it to.
% Then add \input{./title_page_1.tex} to your LaTeX file where you want your
% title page.
%
%%%%%%%%%%%%%%%%%%%%%%%%%%%%%%%%%%%%%%%%%
%\title{Title page with logo}
%----------------------------------------------------------------------------------------
%   PACKAGES AND OTHER DOCUMENT CONFIGURATIONS
%----------------------------------------------------------------------------------------

\documentclass[12pt]{article}
\usepackage[german]{babel}
\usepackage[T1]{fontenc} % for typewriter braces
\usepackage[utf8x]{inputenc}
\usepackage{amsmath}
\usepackage{graphicx}
\usepackage[colorinlistoftodos]{todonotes}
% Simple directory trees
\usepackage{dirtree}
\usepackage{nameref}

% \meta command
\usepackage{textcomp}
\usepackage{hyperref}
\providecommand\meta[1]{\textlangle{\itshape #1\/}\textrangle}

\begin{document}

    \begin{titlepage}

        \newcommand{\HRule}{\rule{\linewidth}{0.5mm}} % Defines a new command for the horizontal lines, change thickness here

        \center % Center everything on page

        % Heading
        \textsc{\LARGE GFOS Innovationsaward 2018}\\[1.5cm]
        \textsc{\Large Helmholtz-Gymnasium Essen}\\[0.5cm]
        \textsc{\large Dokumentation}\\[0.5cm]

        % Main title
        \HRule \\[0.4cm]
        { \huge \bfseries ZNews Server\\ Bedienungsanleitung}\\[0.4cm]
        \HRule \\[1.5cm]

        % Authors
        \begin{minipage}{0.4\textwidth}
            \begin{flushleft}
                \large
                \emph{Autoren:}\\
                Julian Koch\\
                Mats Holtbecker
            \end{flushleft}
        \end{minipage}
        ~
        \begin{minipage}{0.4\textwidth}
            \begin{flushright}
                \large
                \emph{Aufsicht:} \\
                Dennis Gro"skamp\\
                \hfill % Empty line
            \end{flushright}
        \end{minipage}\\[2cm]

        % If you don't want a supervisor, uncomment the two lines below and remove the section above
        %\Large \emph{Author:}\\
        %John \textsc{Smith}\\[3cm] % Your name

        % Date
        {\large \today}\\[2cm]

        % GFOS Logo
        \includegraphics[scale=0.7]{../gfos_logo_edited.jpg}\\[1cm] % Include a department/university logo - this will require the graphicx package

        %----------------------------------------------------------------------------------------

        \vfill
        % Fill the rest of the page with whitespace

    \end{titlepage}


    %% We don't need an abstract in this page
    %% We'll add that to Dokumentation.pdf
    %\begin{abstract}
    %    ZNews besteht weitestgehend aus zwei sich gegen"uberstehenden
    %    Komponenten: dem Server und dem Webinterface.
    %    In diesem Dokument wird die Funktionsweise des Servers
    %    dargestellt und erl"autert.
    %\end{abstract}

    \newpage
    \tableofcontents
    \newpage

    \section{Installation}

    ZNews Server wird als archivierte Datei geliefert.
    Nach deren Download kann sie mit einem Archivierungsprogramm
    wie WinRAR oder 7-Zip ausgepackt werden.

    Die Struktur des Hauptordners ist in Abb.~\ref{fig:dirtree1}
    dargestellt.

    \begin{enumerate}
        \item Extrahieren Sie nach deren Download die Datei \emph{ZNews.zip}.
              Die Struktur des Hauptordners ist in Abb.~\ref{fig:dirtree1} dargestellt.
        \item Starten Sie den Server mit dem Startskript\footnote{%
              Windows: \emph{start.bat}, Mac: \emph{start.command}}.
        \item Nach Erscheinen der Nachricht ``Server started'' (o."a.),
              Schreiben Sie ``end'' und dr"ucken Sie \meta{Enter}.
        \item Nach Beenden des Programms sind die in Abb.~\ref{fig:dirtree2}
              gezeigten Dateien und Ordner neu erstellt worden.
              Konfigurieren Sie den Server wie in Abschnitt~\ref{sec:config} ``\nameref{sec:config}'' beschrieben.
        \item Sie k"onnen nun jederzeit den Server mit dem Startskript starten.
    \end{enumerate}

    \begin{figure}[htb]
        \dirtree{%
        .1 /.
        .2 server.jar.
        .2 start.bat.
        .2 start.command.
        .2 docs/.
        .3 de/.
        .4 Bedienungsanleitung.pdf.
        .4 Server Bedienungsanleitung.pdf.
        .4 Dokumentation.pdf.
        }
        \caption{\label{fig:dirtree1}Aufbau des heruntergeladenen Archivs}
    \end{figure}

    \begin{figure}[htb]
        \dirtree{%
        .1 /.
        .2 logs/.
        .2 static\textunderscore web/.
        .2 auth.json.
        .2 config.properties.
        .2 newsletters.json.
        .2 registrations.json.
        }
        \caption{\label{fig:dirtree2}Neu erstellte Dateien und Ordner}
    \end{figure}

    \section{Konfiguration}%
    \label{sec:config}

    ha







    %Comments can be added to the margins of the document using the \todo{Here's a comment in the margin!} todo command, as shown in the example on the right. You can also add inline comments too:
    %
    %\todo[inline, color=green!40]{This is an inline comment.}


\end{document}