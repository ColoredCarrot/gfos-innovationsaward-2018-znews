Wir haben uns ganz zu Anfang dazu entschieden,
einen eigenen Webserver zu implementieren.
Dabei nutzen wir Netty als eine robuste low-level API.

Unser Webserver kann Anfragen grundlegend auf zwei verschiedene Arten beantworten:
erst geht er eine Liste von registrierten Ressourcen
(im Package \meta{de.znews.server.resources})
durch,
welche alle eine assoziierte URL besitzen
(es besteht die Möglichkeit,
Wildcards in die URL einzufügen,
was URL-Parameter ermöglicht).
Wurde keine passende Ressource gefunden,
wird die Klasse \meta{StaticWeb} genutzt,
welche statische Web-Anfragen beantwortet,
indem sie diese an einen Cache und letztlich das File-System delegiert.

Den genutzten Cache haben wir ebenfalls selbst in der Klasse \meta{SoftLRUCache} programmiert.
Die Namensgebung orientierte sich an folgenden Prinzipien,
die wir bei der Implementierung genutzt haben:
\emph{Soft}, weil die Werte in \emph{SoftReference}s eingeschlossen (\emph{``wrapped''}) sind, und
\emph{LRU} für \emph{least-recently-used}, weil die als letztes genutzten Werte verworfen werden.
Hierbei war es von großer Wichtigkeit,
die Klasse \emph{thread-safe} zu machen,
sodass sie von mehreren konkurrent laufenden Threads gleichzeitig genutzt werden kann;
ich (Julian) habe mich in diesem Kontext oft auf das Buch ``Java Concurrency in Practice''
von Briant Goetz verlassen.
