Der gesamte Arbeitsprozess kann auf dem GitHub-Repo unter
\url{https://github.com/ColoredCarrot/gfos-innovationsaward-2018-znews/}
nachvollzogen werden.

Wir haben uns ganz zu Anfang dazu entschieden,
da noch keiner von uns mit Java EE/Spring gearbeitet hatte,
einfach einen eigenen Webserver zu implementieren.
Dabei nutzen wir Netty als eine robuste low-level API.

Unser WebServer kann Anfragen grundlegend auf zwei verschiedene Arten beantworten:
erst geht er eine Liste von registrierten Ressourcen
(im Package \meta{de.znews.server.resources})
durch,
welche alle eine assoziierte URL besitzen
(es besteht die Möglichkeit,
Wildcards in die URL einzufügen,
was URL-Parameter ermöglicht).
Wurde keine passende Ressource gefunden,
wird die Klasse \meta{StaticWeb} genutzt,
welche statische Web-Anfragen beantwortet,
indem sie diese an einen Cache und letztlich das File-System delegiert.


