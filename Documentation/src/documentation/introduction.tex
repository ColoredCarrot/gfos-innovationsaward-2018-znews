Ein Newsletter-System zu designen und implementieren ist
eine anspruchsvolle Aufgabe,
gerade wegen der nicht sehr spezifischen Anforderungen.
Dies hat uns jedoch viele Freiheiten gelassen,
welche wir auch in vollem Umfang ausgenutzt haben.

Unser Arbeitsprozess lässt sich grob in drei Phasen aufteilen.
Zu Anfang haben wir fast ausschließlich am Webserver und den zugehörigen
Unterprojekten gearbeitet,
also an den Modulen ``ZNews Server'', ``JsonApi'' und ``LoggingApi''.
Die JsonApi habe ich (Julian) schon im letzten Jahr
(dem GFOS Innovationsaward 2017) verwendet,
jedoch für dieses Projekt bis auf den Parser komplett neu geschrieben,
um die Nutzerfreundlichkeit zu erhöhen und weitere Funktionen hinzuzufügen.
Dabei war die bereits bestehende Erfahrung natürlich von Vorteil.
Die LoggingApi haben wir ebenfalls von Grund auf implementiert;
auf sie und die JsonApi wird noch im Detail eingegangen.

Bei dem Webserver selbst haben wir uns für Netty entschieden,
eine \emph{low-level}, robuste, schnelle Bibliothek,
der wir die Implementation des HTTP-Protokolls überließen.
Netty weist jeder Verbindung eine \emph{Pipeline} zu,
welche sog. \emph{Handler} enthält,
die Anfragen in die eine und/oder andere Richtung bearbeiten.
So wandelt der von Netty bereitgestellte \emph{HttpServerCodec}
die rohen eingehenden Bytes in verständlichere Objekte um
(und andersherum für ausgehende Anfragen),
und unser \emph{ResourceProviderHander} verarbeitet die Anfragen
und sucht nach passenden Ressourcen bzw. ruft \emph{StaticWeb} auf.

Weiter ging es mit dem Modul ``StaticWeb'':
Die grundlegende Struktur des Java-Servers war fertig,
und wir konnten anfangen,
das Webinterface zu erstellen.
Sehr früh haben wir uns dazu entschieden,
die Bibliothek \emph{Materialize}\footnote{\url{http://materializecss.com/}}
zu nutzen,
eine Alternative zu dem allbekannten \emph{Bootstrap}-Framework,
welche Websites grundlegend in ein \emph{Grid-System} einteilt
und viele nützliche Elemente wie sog. Karten \textemdash{}
hervorgehobene kleine Sektionen der Seite optional mit Titel und Links im unteren Bereich \textemdash{}
bereitstellt.

Natürlich war zu diesem Zeitpunkt die erste Phase auch noch nicht komplett abgeschlossen;
im Laufe der weiteren Entwicklung mussten wir oft weitere Ressourcen hinzufügen
und bereits bestehende Klassen modifizieren
(beispielsweise um das Bearbeiten der E-Mail-Adresse einer Registrierung zu ermöglichen).

Die letzte Phase war die Entwicklung der Dokumentation,
also dieses Dokuments und der beiden Bedienungsanleitungen.
Im Gegensatz zum letzten Jahr haben wir die Texte nicht in einem WYSIWYG-Editor wie Word verfasst,
sondern in \LaTeX{} geschrieben.
Dieses Dokumenten-Präparationssystem separiert grundlegend das Layout der fertigen PDF
von dem eigentlichen Inhalt,
was sehr nützlich ist,
da das Layout erst am Ende des Arbeitsprozesses festgelegt werden kann.
Des Weiteren kann das Dokument in mehrere Dateien aufgeteilt werden,
was die Übersicht des Projektes bedeutend erhöht.

Unser gesamter Arbeitsprozess kann auf dem GitHub-Repo unter
\url{https://github.com/ColoredCarrot/gfos-innovationsaward-2018-znews/}
nachvollzogen werden.
