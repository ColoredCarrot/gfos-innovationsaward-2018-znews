ZNews ist in fünf Hauptprojekte aufgeteilt,
welche im Folgenden genannt und deren Funktionen beschrieben werden.
Die Namen der Unterprojekte sind von der Struktur des IntelliJ-Projektes übernommen.

\begin{itemize}
    \item \textbf{ZNews Server}
    beinhaltet den Java-Code unseres Webservers,
    der Ressourcen, die die API dessen bilden,
    des Modells der Newsletter/Statistiken/etc.,
    der Serialisierungsschnittstelle
    und weiterer Utility-Funktionen.

    \item \textbf{StaticWeb}
    umfasst die vom Server versendbaren Dateien,
    inklusive aller HTML-, CSS- und JavaScript-Dateien und Bilder.
    %Im Ordner \emph{admin} befinden sich die nur von Administratoren erreichbaren Ressourcen

    \item \textbf{LoggingApi}
    ist die von ZNews genutzte Logging-Schnittstelle,
    welche wir vollständig selbst geschrieben haben.
    Sie kann \meta{LogRecord}s,
    welche den ausführenden Thread, den zugehörigen StackTrace,
    die Nachricht selbst und weitere Informationen beinhalten,
    mit verschiedenen \meta{Logger}n bearbeiten.
    Zum Beispiel ist der \meta{LevelFilterLogger} dafür verantwortlich,
    Records, die ein bestimmtes \meta{LogRecord.Level} unterschreiten,
    herauszufiltern;
    der \meta{NewThreadLogger} delegiert alle Logging-Anfragen an einen weiteren Thread,
    der \meta{StringFormatLogger} formatiert alle Nachrichten
    (ersetzt Variablen wie ``{threadName}'' mit dem entsprechenden Wert);
    und letztlich speichert der \meta{FilesGzipLogger} die Records in einer GZIP-komprimierten Datei.

    \item \textbf{JsonApi}
    schließt unsere JSON-Schnittstelle ein.
    Die API erlaubt das Konvertieren von Text im JSON-Format
    in einen AST (Abstract Syntax Tree, im Groben eine baum'sche Repräsentation der Daten)
    und weiter in ein POJO (Plain Old Java Object).
    Dazu nutzt sie unseren eigenen JSON-Parser,
    welcher aus einem Character-Stream einen Stream von \meta{JsonToken}s,
    also nützlicherer ``Stücke'' des Inputs
    wie einer geöffneten Klammer oder eines in Anführungszeichen eingeschlossenen Strings,
    generiert
    und diesen in der Klasse \meta{JsonParser} in den bereits erwähnten AST übersetzt.

    Um einen AST in ein Objekt umzuwandeln,
    nutzen wir die Java Reflection API
    und, sollten wir keinen passenden Konstruktor in der gewünschten Klasse finden,
    die \emph{Unsafe}-Klasse, die es erlaubt,
    einem Objekt Speicherplatz zuzuweisen,
    ohne einen der Konstruktoren aufzurufen\footnote{%
    Dies ist zwar \emph{dirty},
    jedoch ermöglicht dies eine Nutzerfreundlichkeit,
    die andernfalls nicht zu erreichen wäre
    (so muss der Nutzer nicht explizit einen \emph{no-args}-Konstruktor für jede serialisierbare Klasse erstellen)}.

    \item \textbf{Documentation}
    beinhaltet die gesamte Dokumentation des Projekts
    (ausschließlich der JavaDocs),
    inklusive dieser Datei und der Bedienungsanleitungen des Servers und der Website.

    \item \textbf{Licenses}
    fast die Lizenzen aller genutzten Bibliotheken auf.
\end{itemize}
