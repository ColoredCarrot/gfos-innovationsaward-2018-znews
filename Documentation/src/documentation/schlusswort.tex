Da die Nutzung von PHP untersagt war
(und wir nicht genug Zeit hatten,
eine eigene Programmiersprache zu schreiben \smiley ),
wird fast keine HTML-Seite vom Server generiert;
stattdessen nutzen wir JavaScript und AJAX,
um dynamischen Content zu laden.
Dies ist häufig jedoch sehr mühselig,
und hat letztlich einen bleibenden Eindruck
der Wichtigkeit einer Template-Sprache wie PHP
hinterlassen.

Trotzdem hat mir (Julian) persönlich die Arbeit an dem Projekt sehr viel Spaß bereitet.
Auch wenn Web-Design nicht zu meinen Lieblings-Beschäftigungen zählt,
so bin ich doch glücklich mit unserer finalen Version.
Ich konnte weiter Erfahrungen über die Implementation von Servern sammeln \textemdash{}
nicht zuletzt, welch hochkomplexe Systeme Server wie Tomcat eigentlich sind.
