E-Mails zu versenden gehörte auch schon letztes Jahr zur Aufgabenstellung
und hat uns dieses Jahr mindestens ebensoviele Kopfschmerzen bereitet.
Einen Großteil der Zeit (bis zum 18.03.2018)
wiesen alle E-Mails grobe Fehler auf,
die wir uns nicht erklären konnten.

Erst viel später haben wir bemerkt,
dass das Problem nicht etwa bei der Nutzung der JavaMail API lag,
sondern bei unserer internen \meta{Str}-Klasse,
welche wir als mutablen,
effizienter implementierten String vorgesehen hatten;
ein Fehler fürte dazu,
dass u.A. die \meta{replace}-Methode den String auf unvorhergesehene Weise verändert hat.

Das beschriebene Problem bestand darin,
dass die von \meta{replace} genutzte Methode \meta{shift},
welche Charactere um eine Anzahl an ``Plätzen'' verschiebt,
nicht die Länge des \meta{Str}s angepasst hat.

Jedoch war dies nicht die einzige Schwierigkeit:
das Generieren des HTML-Codes erwies sich als besonders kompliziert,
da unser RichText-Editor die Bibliothek Quill nutzt,
welche zum Speichern des Textes das hauseigene, JSON-basierte \emph{Delta}-Format nutzt.
Dieses Format kann nur Quill selbst rendern,
weshalb der Server diese Bibliothek hätte aufrufen müssen.
Durch JavaScript-Script-Engines wie Nashorn wäre dies zwar möglich,
jedoch,
aufgrund der Dependency von Quill auf jQuery und damit das DOM (Document Object Model),
extrem zeitaufwendig und letzten Endes wahrscheinlich nicht die hohe Investition wert.
So haben wir uns schließlich entschieden,
den Newsletter nicht in der E-Mail zu rendern und stattdessen einen direkten Link einzufügen.
