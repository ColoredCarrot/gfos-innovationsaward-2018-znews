\emph{Fair warning: In diesem Teil der Dokumentation nutzen wir Witze,
um nicht aufgrund der Sensibilität des Themas unseren Kopf zu verlieren.}

Die Aufgabenstellung verlangte \emph{``nur rudimentäre''} Möglichkeiten zur Bearbeitung der Artikel
und nannte als Beispiel einen RichText-Editor.
\emph{Hahaha. Haha. Ha.}

Spaß beiseite,
aber es ist tatsächlich unglaublich schwierig,
einen ausreichend vielfältigen, robusten, dynamischen, kompatiblen
Editor zu finden.
Dies liegt nicht zuletzt DEM Horror des HTML5-Standards, \emph{contentEditable}.
(siehe \href{https://medium.engineering/why-contenteditable-is-terrible-122d8a40e480}%
{das erste relevante Suchergebnis auf Google}\footnote{\url{https://medium.engineering/why-contenteditable-is-terrible-122d8a40e480}}.)

Zu Testzwecken und um zu prokrastinieren haben wir die Entscheidung erstmal verlegt
und einen Markdown-Editor implementiert.
Jedoch waren wir hiermit auf lange Sicht nicht zufrieden,
weßhalb wir uns nach \emph{langer} Recherche für den RichText-Editor \emph{Quill} entschieden haben.,
nicht zuletzt aufgrund dessen Entscheidung,
das DOM ``links liegen zu lassen'' und stattdessen ein eigenes Modell,
\emph{Parchment}\footnote{\url{https://quilljs.com/guides/cloning-medium-with-parchment/}},
zu nutzen.
