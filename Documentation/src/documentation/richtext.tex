Die Aufgabenstellung verlangte \emph{``nur rudimentäre''}
Möglichkeiten zur Bearbeitung der Artikel
und nannte als Beispiel einen RichText-Editor.

Wie sich jedoch herausstellte,
war diese als ``rudimentär'' bezeichnete Option
alles andere als einfach.
Wie jedem in dem Thema drinnen auffallen dürfte,
ist es tatsächlich unglaublich schwierig,
einen ausreichend vielfältigen, robusten, dynamischen, kompatiblen
RichText-Editor zu finden.
Dies liegt nicht zuletzt an \emph{contentEditable}, dem HTML5-Standard,
welcher von fast allen Seiten als insuffizient
zur Implementierung eines simplen Editors gesehen wird.
(siehe \href{https://medium.engineering/why-contenteditable-is-terrible-122d8a40e480}%
{das erste relevante Suchergebnis auf Google}\footnote{\url{https://medium.engineering/why-contenteditable-is-terrible-122d8a40e480}}.)

Zu Testzwecken und um zu prokrastinieren haben wir die Entscheidung erstmal verlegt
und einen Markdown-Editor implementiert.
Jedoch waren wir hiermit auf lange Sicht nicht zufrieden,
weshalb wir uns nach \emph{langer} Recherche für den RichText-Editor \emph{Quill} entschieden haben,
nicht zuletzt aufgrund dessen Entscheidung,
das DOM ``links liegen zu lassen'' und stattdessen ein eigenes Modell,
\emph{Parchment}\footnote{\url{https://quilljs.com/guides/cloning-medium-with-parchment/}},
zu nutzen.
