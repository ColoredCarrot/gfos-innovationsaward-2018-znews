Die Aufgabenstellung verlangte \emph{``nur rudimentäre''}
Möglichkeiten zur Bearbeitung der Artikel
und nannte als Beispiel einen RichText-Editor.

Wie sich jedoch herausstellte,
war diese als ``rudimentär'' bezeichnete Option
alles andere als einfach.
Jedem,
der sich intensiv mit diesem Thema auseinandersetzt,
fällt auf,
dass es tatsächlich unglaublich schwierig ist,
einen ausreichend vielfältigen, robusten, dynamischen, kompatiblen
RichText-Editor zu finden.
Dies liegt nicht zuletzt an \emph{contentEditable}, dem einzigen hierfür verwendbaren HTML5-Standard,
welcher jedoch von fast allen Seiten als insuffizient
zur Implementierung eines simplen Editors gesehen wird.
(siehe \href{https://medium.engineering/why-contenteditable-is-terrible-122d8a40e480}%
{das erste relevante Suchergebnis auf Google}\footnote{\url{https://medium.engineering/why-contenteditable-is-terrible-122d8a40e480}}.)

Zu Testzwecken und um zu prokrastinieren haben wir die Entscheidung erstmal verschoben
und einen Markdown-Editor implementiert.
Jedoch waren wir hiermit auf lange Sicht nicht zufrieden,
weshalb wir uns nach intensiver Recherche für den RichText-Editor \emph{Quill} entschieden haben,
nicht zuletzt, weil Quill
das DOM ``links liegen lässt'' und stattdessen ein eigenes Modell,
\emph{Parchment}\footnote{\url{https://quilljs.com/guides/cloning-medium-with-parchment/}},
nutzt.
