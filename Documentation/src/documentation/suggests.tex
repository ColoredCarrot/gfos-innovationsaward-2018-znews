Ein Problem, das ein \emph{hard limit} auf die Skalierbarkeit von ZNews setzt,
ist, dass in der aktuellen Version mit jedem Aufruf der Startseite der Client automatisch
\emph{alle} Artikel herunterläd.
Bei wenigen Newslettern ist dies zwar kaum bemerkbar,
jedoch vergrößert dies die Ladezeit mit jedem neuen Artikel,
speziell, wenn diese sehr lang sind oder Bilder beinhalten.

Die Anzahl der herunterzuladenden Artikel zu begrenzen ist hier eine offensichtliche Option;
ein ``Load More''-Button könnte weitere Artikel nachladen.
Eine weitere Möglichkeit ist,
jeden Artikel erst bei einem Klick auf die entsprechende Überschrift zu laden,
jedoch würde dies die Reaktionszeit (``response time'') erheblich erhöhen.

Des Weiteren besitzt die JsonApi kein \emph{type-binding}-System\footnote{%
Siehe \href{https://github.com/ColoredCarrot/gfos-innovationsaward-2018-znews/issues/19}{Issue #19 des GitHub-Projekts}},
es besteht also keine Möglichkeit,
Typen, deren Source-Code man nicht verändern kann,
zu (de)serialisieren,
sollte die eingebaute Reflection und Unsafe-gestützte Lösung nicht ausreichen.
Zum Beispiel war es nötig,
die Serialisierung von \emph{Date}s fest im Quellcode der API einzubinden,
welche nur den Timestamp (die Anzahl der Millisekunden seit dem 01.01.1970, Mitternacht, UTC)
benötigt.
